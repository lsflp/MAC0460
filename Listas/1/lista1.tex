\documentclass[12pt,letterpaper]{article}
\usepackage[utf8]{inputenc}
\usepackage{amsmath,amsthm,amsfonts,amssymb,amscd}
\usepackage[table]{xcolor}
\usepackage[margin=2cm]{geometry}
\usepackage{graphicx}
\usepackage{multicol}
\usepackage{mathtools}
\usepackage{palatino}
\usepackage[brazil]{babel}
\newlength{\tabcont}
\setlength{\parindent}{0.0in}
\setlength{\parskip}{0.05in}

\begin{document}
	
	\textbf{Nome}: Luís Felipe de Melo Costa Silva \\
	\textbf{Número USP}: 9297961 
	
	\begin{center}
		\LARGE \bf
		Lista de Exercícios 2 - MAC0460
	\end{center}
	
	\section*{Exercício 1}
	
	\textbf{a)} Vamos fazer $w^T(t) \cdot x(t) = y^*(t)$. Como $x(t)$ está classificado incorretamente, temos que os sinais de $y(t)$ e $y^*(t)$ são diferentes, ou seja, ou $y(t) = 1$ e $y^*(t) = -1$ ou $y(t) = -1$ e $y^*(t) = 1$. Portanto, $y(t) \cdot y^*(t)$ é sempre $-1$ e então, $y(t) \cdot y^*(t) < 0$
	
	\textbf{b)} Usando (1.3):
	
	\begin{equation*}
		\begin{split}
			y(t)w^T(t+1)x(t) > y(t)w^T(t)x(t) \\
			y(t)[w(t) + y(t)x(t)]^Tx(t) > y(t)w^T(t)x(t) \\
			y(t)[w^T(t) + [y(t)x(t)]^T]x(t) > y(t)w^T(t)x(t) \\
			y(t)w^T(t)x(t) + y(t)[y(t)x(t)]^Tx(t) > y(t)w^T(t)x(t) \\
			y(t)[y(t)x(t)]^Tx(t) > 0 \\
		\end{split}
	\end{equation*}
	
	\section*{Exercício 4}
	
	Queremos que $\epsilon(M, N, \delta) = \sqrt{\frac{1}{2N} \text{ln} \frac{2M}{\delta}} \leq 0.05$. Teremos:
	
	\begin{multicols}{2}
		\begin{equation*}
			\begin{split}
				\sqrt{\frac{1}{2N} \text{ln} \frac{2M}{\delta}} \leq 0.05 \\
				\frac{1}{2N} \text{ln} \frac{2M}{\delta} \leq 0.05^2 \\
				\frac{1}{2N} \leq \frac{0.05^2}{\text{ln} \frac{2M}{\delta}} \\
				2N \leq \frac{\text{ln} \frac{2M}{\delta}}{0.05^2} \\
				N \leq \frac{\text{ln} \frac{2M}{\delta}}{2 \cdot 0.05^2} \\
			\end{split}
		\end{equation*}
		
		Como $\delta = 0.03$: \\
		
		\textbf{a)} Para $M = 1$, $N \leq \frac{\text{ln} \frac{2}{0.03}}{2 \cdot 0.05^2} \cong 840$
		
		\textbf{b)} Para $M = 100$, $N \leq \frac{\text{ln} \frac{200}{0.03}}{2 \cdot 0.05^2} \cong 1761$
		
		\textbf{c)} Para $M = 10000$, $N \leq \frac{\text{ln} \frac{20000}{0.03}}{2 \cdot 0.05^2} \cong 2683$
		
	\end{multicols}
	
	
	
\end{document}